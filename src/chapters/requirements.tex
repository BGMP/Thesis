\chapter{Requerimientos del Software}

\section{Límites}

\begin{itemize}
	\item El software no permitirá relacionar automáticamente cualquier nombre de jugador con su perfil de usuario registrado en ella.
	\item El software no registrará estadísticas de manera retroactiva para jugadores no registrados.
	\item El software no permitirá visualizar carreras de manera individual.
	\item Las estadísticas por jugador no pueden ser editadas manualmente, sino que la modificación de estas debe ser un efecto de subir o eliminar sesiones.
	\item Las sesiones eliminadas no se podrán visualizar. Esto quiere decir que las sesiones que se eliminen no podrán ser accedidas nunca más una vez borradas.
	\item Los usuarios no podrán cancelar sus cuentas por si mismos.
	\item El software no permitirá editar los modelos de sesiones de manera manual.
\end{itemize}

\section{Caracterización de los Usuarios}
Como se ha mencionado a lo largo de este informe, los usuarios a quienes apunta este proyecto de título son aquellos que forman parte de la comunidad de Re-Volt America. Estos usuarios, de manera general, se caracterizan por:

\begin{itemize}
	\item Pertenencia a un grupo etario entre 14 y 26 años.
	\item Competencias técnicas en uso de software similares.
	\item Familiaridad con el juego y el sistema  de puntos de RVA.
\end{itemize}

Por otra parte, tenemos la definición de quienes pueden ser clasificados como usuarios del software, así como la frecuencia con la que utilizarán la plataforma y su rol dentro de la misma:

\begin{center}
	\begin{tabular}{| p{3cm} | p{3cm} | p{10cm} |}
		\hline
		\multicolumn{3}{|c|}{\textbf{Usuarios}} \\
		\hline
		\multicolumn{1}{|c|}{\textbf{Rol}} & \multicolumn{1}{|c|}{\textbf{Nivel}} & \multicolumn{1}{|c|}{\textbf{Descripción}} \\
		\hline
		{\textbf{Administrador}} & Primario & Utiliza todas las funciones de la página, como subir autos, pistas, sesiones y manejo general del software. Configura el software para el resto de los usuarios.\\ \hline
		{\textbf{Organizador}} & Primario & Utiliza las funciones específicamente relacionadas a la subida y manejo de sesiones en la página.\\ \hline
		{\textbf{Moderador}} & Secundario & Utiliza las funciones específicas de manejo de usuarios, tal como la edición de perfiles o aplicación de infracciones. \\ \hline
		{\textbf{Jugador}} & Terciario & Utiliza la página sólo para visualizar la información que esta ofrece. \\ \hline
	\end{tabular}
  
  \captionof{table}{Tabla de tipos de usuario}\label{table:users:type}
\end{center}

\section{Objetivo General del Software}
El sistema manejará información del proceso de cálculo de resultados de carreras para que la comunidad centralice los rankings y estadísticas por usuario dentro del mismo, es decir, reducirá el trabajo manual requerido actualmente para este procesamiento de resultados, y así hará más eficiente todo el proceso que conlleva mantener los rankings actualizados.

\subsection{Objetivos Específicos del Software}
\begin{itemize}
	\item El sistema permite que los administradores de la comunidad puedan subir los archivos de resultados generados por RVGL y, de esta forma, los resultados son generados automáticamente dentro de la aplicación. Esto elimina el tiempo de los organizadores de calcular los resultados manualmente.
	\item El sistema permite que los usuarios puedan ver sus estadísticas en tiempo real, lo que elimina la necesidad de llevar la cuenta de manera manual por cada uno de ellos.
	\item El sistema permite enlazar los nombres de usuario utilizados en las sesiones multijugador de RVGL a perfiles dentro de la aplicación, lo cual hace posible la recopilación y atribución de métricas individuales por jugador, y agiliza la visualización de resultados.
	\item El sistema permite llevar un registro histórico de manera automática según se suben y se procesan los archivos de resultados en la aplicación, lo cual elimina completamente la necesidad de la comunidad de mantener toda esta información actualizada de manera manual.
\end{itemize}

\section{Requerimientos Funcionales del Software}
% Autos
\begin{center}
	\begin{tabular}{ | l | p{15cm} |}
		\hline
		\multicolumn{2}{|c|}{\textbf{Módulo de Registro de Autos de Re-Volt America}} \\
		\hline
		\multicolumn{1}{|c|}{\textbf{Id}} & \multicolumn{1}{|c|}{\textbf{Descripción}} \\
		\hline
		{\textbf{RF\_01}} & La plataforma contará con un módulo de creación de autos. Dentro de este módulo, los autos deben ser relacionados con una temporada. No podrán existir autos con nombres duplicados dentro de la misma temporada. Los autos deben ser importados desde un archivo separado por comas, el cual debe contar con todos los parámetros obligatorios del modelo de auto. Sólo los administradores pueden crear autos. \\ \hline

		{\textbf{RF\_02}} & La plataforma contará con un módulo de visualización de autos. El listado estará separado por temporadas y por clases de autos. Este módulo estará disponible para cualquier tipo de usuario. \\ \hline

		{\textbf{RF\_03}} & La plataforma contará con un módulo de edición de un auto. Los autos deben contar con todos los parámetros obligatorios para ser modificados. Este módulo estará disponible sólo para los administradores. \\ \hline
		
		{\textbf{RF\_04}} & La plataforma contará con un módulo de eliminación de un auto. La eliminación no tendrá efectos secundarios a nivel de la base de datos, ya que la aplicación estará preparada para manejar excepciones cuando las entradas de corredores estén asociadas a un auto que no exista. Este módulo sólo estará disponible para administradores. \\ \hline
		
		{\textbf{RF\_05}} & La plataforma contará con un módulo de visualización de un sólo auto. En esta vista se podrán ver todos los parámetros del auto. Este módulo estará disponible para cualquier tipo de usuario. \\ \hline
	\end{tabular}
  
  \captionof{table}{Tabla de requerimientos funcionales del módulo de autos}\label{table:rf:cars}
\end{center}

% Pistas
\begin{center}
	\begin{tabular}{ | l | p{15cm} |}
		\hline
		\multicolumn{2}{|c|}{\textbf{Módulo de Registro de Pistas de Re-Volt America}} \\
		\hline
		\multicolumn{1}{|c|}{\textbf{Id}} & \multicolumn{1}{|c|}{\textbf{Descripción}} \\
		\hline
		{\textbf{RF\_06}} & La plataforma contará con un módulo de creación de pistas. Dentro de este módulo, las pistas deben ser relacionados con una temporada. No podrán existir pistas con nombres duplicados dentro de la misma temporada. Las pistas deben contar con todos los parámetros obligatorios del modelo de pista. Sólo los administradores pueden crear pistas. \\ \hline
		
		{\textbf{RF\_07}} & La plataforma contará con un módulo de visualización de pistas. El listado estará separado por temporadas y paginado. Este módulo estará disponible para cualquier tipo de usuario. \\ \hline
		
		{\textbf{RF\_08}} & La plataforma contará con un módulo de edición de una pistas. Las pistas deben contar con todos los parámetros obligatorios para ser modificadas. Este módulo estará disponible sólo para los administradores. \\ \hline
		
		{\textbf{RF\_09}} & La plataforma contará con un módulo de eliminación de una pista. La eliminación no tendrá efectos secundarios a nivel de la base de datos, ya que la aplicación estará preparada para manejar excepciones cuando las carreras estén asociadas a una pista que no exista. Este módulo sólo estará disponible para administradores. \\ \hline
		
		{\textbf{RF\_10}} & La plataforma contará con un módulo de visualización de una sola pista. En esta vista se podrán ver todos los parámetros de la pista. Este módulo estará disponible para cualquier tipo de usuario. \\ \hline
	\end{tabular}
  
  \captionof{table}{Tabla de requerimientos funcionales del módulo de pistas}\label{table:rf:tracks}
\end{center}

% Temporadas
\begin{center}
	\begin{tabular}{ | l | p{15cm} |}
		\hline
		\multicolumn{2}{|c|}{\textbf{Módulo de Registro de Temporadas de Re-Volt America}} \\
		\hline
		\multicolumn{1}{|c|}{\textbf{Id}} & \multicolumn{1}{|c|}{\textbf{Descripción}} \\
		\hline
		{\textbf{RF\_11}} & La plataforma contará con un módulo de creación de temporadas. Las temporadas deben contar con todos los parámetros obligatorios. Al momento de crear una temporada, sus 6 rankings asociados deberán crearse automáticamente. Sólo los administradores pueden crear temporadas. \\ \hline
		
		{\textbf{RF\_12}} & La plataforma contará con un módulo de visualización de temporadas. El listado contendrá todas las temporadas, las cuales vendrán ordenadas por fecha de inicio. Este módulo estará disponible para cualquier tipo de usuario. \\ \hline
		
		{\textbf{RF\_13}} & La plataforma contará con un módulo de edición de una temporada, las cuales deben contar con todos los parámetros obligatorios para ser modificados. Este módulo estará disponible sólo para los administradores. \\ \hline
		
		{\textbf{RF\_14}} & La plataforma contará con un módulo de eliminación de una temporada. Al eliminarse una temporada, se eliminarán también todos sus rankings asociados. Al mismo tiempo, todas las sesiones asociadas a dichos rankings también serán eliminadas. Las estadísticas globales de cada jugador también serán substraídas de sus perfiles. Este módulo sólo estará disponible para administradores. \\ \hline
		
		{\textbf{RF\_15}} & La plataforma contará con un módulo de visualización de una sola temporada. En esta vista se podrán ver los rankings asociados y la tabla de resultados de la temporada. Este módulo estará disponible sólo para cualquier tipo de usuario.\\ \hline
	\end{tabular}
  
  \captionof{table}{Tabla de requerimientos funcionales del módulo de temporadas}\label{table:rf:seasons}
\end{center}

% Rankings
\begin{center}
	\begin{tabular}{ | l | p{15cm} |}
		\hline
		\multicolumn{2}{|c|}{\textbf{Módulo de Visualización de Rankings de Re-Volt America}} \\
		\hline
		\multicolumn{1}{|c|}{\textbf{Id}} & \multicolumn{1}{|c|}{\textbf{Descripción}} \\
		\hline
		
		{\textbf{RF\_16}} & La plataforma contará con un módulo de visualización de rankings, el cual se encontrará dentro de cada visualización de temporada. Este módulo estará disponible para cualquier tipo de usuario. \\ \hline
		
		{\textbf{RF\_17}} & La plataforma contará con un módulo de visualización de un solo ranking. En esta vista se podrán ver todas las sesiones asociadas al ranking, junto con su tabla de resultados acumulados. Este módulo estará disponible para cualquier tipo de usuario. \\ \hline
	\end{tabular}
  
  \captionof{table}{Tabla de requerimientos funcionales del módulo de visualización de rankings}\label{table:rf:rankings}
\end{center}

% Sesiones
\begin{center}
	\begin{tabular}{ | l | p{15cm} |}
		\hline
		\multicolumn{2}{|c|}{\textbf{Módulo de Sesiones de Re-Volt America}} \\
		\hline
		\multicolumn{1}{|c|}{\textbf{Id}} & \multicolumn{1}{|c|}{\textbf{Descripción}} \\
		\hline
		{\textbf{RF\_18}} & La plataforma contará con un módulo de subida de sesiones en forma de archivo separado por comas. El archivo separado por comas deberá ser un Session Log generado por RVGL. Este módulo deberá permitir al usuario seleccionar un archivo separado por comas. Las sesiones deben contar con todos los parámetros obligatorios. Sólo los administradores pueden crear sesiones. \\ \hline
		
		{\textbf{RF\_19}} & La plataforma contará con un módulo de visualización de sesiones. El listado contendrá las últimas sesiones del ranking vigente de la temporada actual. Este módulo estará disponible para cualquier tipo de usuario. \\ \hline
		
		{\textbf{RF\_20}} & La plataforma contará con un módulo de eliminación de una sesión. La eliminación gatilla la sustracción de las estadísticas de los usuarios que participaron en ella. Este módulo sólo estará disponible para administradores. \\ \hline
		
		{\textbf{RF\_21}} & La plataforma contará con un módulo de visualización de una sola sesión. En esta vista se podrán ver todos los parámetros de la sesión, ordenados en el formato de RVA. Este módulo estará disponible para cualquier tipo de usuario. \\ \hline
	\end{tabular}
  
  \captionof{table}{Tabla de requerimientos funcionales del módulo de sesiones}\label{table:rf:sessions}
\end{center}

\section{Requerimientos No Funcionales del Software}
La presente sección hablará de los requerimientos no funcionales de la aplicación desarrollada. Todos los requerimientos no funcionales se relacionarán con uno o más atributos. Si un atributo aplica a un requerimiento no funcional, eso quiere decir que el requerimiento contribuye a la calidad del software desarrollado a través de ese atributo. Todos los atributos listados están basados en la norma ISO 25010.

% API
\begin{center}
  \begin{tabular}{ | p{2cm}| p{8cm} | p{5cm} |}
    \hline
    \multicolumn{3}{|c|}{\textbf{RNF\_01}} \\
    \hline
    
    \multicolumn{1}{|p{2cm}|}{\textbf{Descripción}} & \multicolumn{2}{|p{13cm}|}{La plataforma contará con una API REST, la cual estará diseñada para ser consumida por aplicaciones de terceros. Todos los endpoints estarán disponibles sólo para lectura por parte de terceros.} \\ \hline
    
    \multicolumn{1}{|p{3.5cm}|}{\textbf{{Atributo}}} & \multicolumn{1}{|p{1.5cm}|}{\textbf{Aplica}} & \multicolumn{1}{|p{10cm}|}{\textbf{Especificación}} \\ \hline
    
    \multicolumn{1}{|p{3.5cm}|}{\nohyphens{Adecuación Funcional}} & \multicolumn{1}{|c|}{X} & \multicolumn{1}{|p{10cm}|}{La API contribuye a la corrección funcional, ya que facilita la obtención de datos precisos del sistema a terceros.} \\ \hline
    
    \multicolumn{1}{|p{3.5cm}|}{\nohyphens{Eficiencia de Desempeño}} & \multicolumn{1}{|c|}{} & \multicolumn{1}{|p{10cm}|}{} \\ \hline
    
    \multicolumn{1}{|p{3.5cm}|}{\nohyphens{Compatibilidad}} & \multicolumn{1}{|c|}{X} & \multicolumn{1}{|p{10cm}|}{La API contribuye a la coexistencia con otras piezas de software independientes, ya que permite a dicho software consumir información del sistema en tiempo real.} \\ \hline
    
    \multicolumn{1}{|p{3.5cm}|}{\nohyphens{Usabilidad}} & \multicolumn{1}{|c|}{} & \multicolumn{1}{|p{10cm}|}{} \\ \hline
    
    \multicolumn{1}{|p{3.5cm}|}{\nohyphens{Fiabilidad}} & \multicolumn{1}{|c|}{X} & \multicolumn{1}{|p{10cm}|}{La API contribuye a la madurez del software, ya que es gracias a ella que el sistema puede satisfacer las necesidades de los usuarios que consumen información del mismo.} \\ \hline
    
    \multicolumn{1}{|p{3.5cm}|}{\nohyphens{Seguridad}} & \multicolumn{1}{|c|}{X} & \multicolumn{1}{|p{10cm}|}{Gracias al diseño de la API, sólo se exponen endpoints de lectura, por lo que esta contribuye a la confidencialidad e integridad de la información.} \\ \hline
    
    \multicolumn{1}{|p{3.5cm}|}{\nohyphens{Mantenibilidad}} & \multicolumn{1}{|c|}{} & \multicolumn{1}{|p{10cm}|}{} \\ \hline
    
    \multicolumn{1}{|p{3.5cm}|}{\nohyphens{Portabilidad}} & \multicolumn{1}{|c|}{} & \multicolumn{1}{|p{10cm}|}{} \\

    \hline
  \end{tabular}
  
  \captionof{table}{Tabla de requerimiento no funcional de API}\label{table:rnf:api}
\end{center}

% Roles
\begin{center}
  \begin{tabular}{ | p{2cm}| p{8cm} | p{5cm} |}
    \hline
    \multicolumn{3}{|c|}{\textbf{RNF\_02}} \\
    \hline
    
    \multicolumn{1}{|p{2cm}|}{\textbf{Descripción}} & \multicolumn{2}{|p{13cm}|}{La plataforma contará con una separación interna de roles a nivel de software. Estos roles serán: administrador, organizador, moderador y jugador, siendo el último denotado por no tener ninguno de los otros roles. El rol del primer administrador del sistema deberá ser asignado de manera interna. Una vez que existe un administrador, este podrá modificar los roles de todos los usuarios desde la misma página web.} \\ \hline
    
    \multicolumn{1}{|p{3.5cm}|}{\textbf{{Atributo}}} & \multicolumn{1}{|p{1.5cm}|}{\textbf{Aplica}} & \multicolumn{1}{|p{10cm}|}{\textbf{Especificación}} \\ \hline
    
    \multicolumn{1}{|p{3.5cm}|}{\nohyphens{Adecuación Funcional}} & \multicolumn{1}{|c|}{X} & \multicolumn{1}{|p{10cm}|}{Contribuye a la completitud y pertinencia funcional, ya que, gracias a contar con una separación de privilegios de usuario, el sistema puede proveerles opciones específicas basado en la jerarquía de roles definida.} \\ \hline
    
    \multicolumn{1}{|p{3.5cm}|}{\nohyphens{Eficiencia de Desempeño}} & \multicolumn{1}{|c|}{} & \multicolumn{1}{|p{10cm}|}{} \\ \hline
    
    \multicolumn{1}{|p{3.5cm}|}{\nohyphens{Compatibilidad}} & \multicolumn{1}{|c|}{} & \multicolumn{1}{|p{10cm}|}{} \\ \hline
    
    \multicolumn{1}{|p{3.5cm}|}{\nohyphens{Usabilidad}} & \multicolumn{1}{|c|}{} & \multicolumn{1}{|p{10cm}|}{} \\ \hline
    
    \multicolumn{1}{|p{3.5cm}|}{\nohyphens{Fiabilidad}} & \multicolumn{1}{|c|}{} & \multicolumn{1}{|p{10cm}|}{} \\ \hline
    
    \multicolumn{1}{|p{3.5cm}|}{\nohyphens{Seguridad}} & \multicolumn{1}{|c|}{X} & \multicolumn{1}{|p{10cm}|}{La presencia de roles en el sistema contribuye directamente a la autenticidad, ya que con ellos el sistema es capaz de decernir si el usuario debiera poder realizar una determinada acción o no basándose en su identidad dentro del software.} \\ \hline
    
    \multicolumn{1}{|p{3.5cm}|}{\nohyphens{Mantenibilidad}} & \multicolumn{1}{|c|}{} & \multicolumn{1}{|p{10cm}|}{} \\ \hline
    
    \multicolumn{1}{|p{3.5cm}|}{\nohyphens{Portabilidad}} & \multicolumn{1}{|c|}{} & \multicolumn{1}{|p{10cm}|}{} \\
    
    \hline
  \end{tabular}
  
  \captionof{table}{Tabla de requerimiento no funcional de roles de usuario}\label{table:rnf:roles}
\end{center}

% Roles
\begin{center}
  \begin{tabular}{ | p{2cm}| p{8cm} | p{5cm} |}
    \hline
    \multicolumn{3}{|c|}{\textbf{RNF\_03}} \\
    \hline
    
    \multicolumn{1}{|p{2cm}|}{\textbf{Descripción}} & \multicolumn{2}{|p{13cm}|}{El código estará versionado a través de Git, utilizando GitHub como plataforma de almacenamiento en la nube para el control de versiones. El proyecto será de código abierto, y se contará con un sistema de integración continua para el software implementado directamente en GitHub. A través de este sistema se podrá ser desplegar el software de manera automática cuando se detecten cambios en una rama de Git determinada.} \\ \hline
    
    \multicolumn{1}{|p{3.5cm}|}{\textbf{{Atributo}}} & \multicolumn{1}{|p{1.5cm}|}{\textbf{Aplica}} & \multicolumn{1}{|p{10cm}|}{\textbf{Especificación}} \\ \hline
    
    \multicolumn{1}{|p{3.5cm}|}{\nohyphens{Adecuación Funcional}} & \multicolumn{1}{|c|}{} & \multicolumn{1}{|p{10cm}|}{} \\ \hline
    
    \multicolumn{1}{|p{3.5cm}|}{\nohyphens{Eficiencia de Desempeño}} & \multicolumn{1}{|c|}{} & \multicolumn{1}{|p{10cm}|}{} \\ \hline
    
    \multicolumn{1}{|p{3.5cm}|}{\nohyphens{Compatibilidad}} & \multicolumn{1}{|c|}{X} & \multicolumn{1}{|p{10cm}|}{Contribuye a la coexistencia del software con otros sistemas, ya que, gracias a estar versionado en Git y ser de código abierto, los mantenedores y desarrolladores de otras piezas de software podrán tener acceso a toda la información técnica que necesiten para poder integrar sus productos con la plataforma de RVA exitosamente.} \\ \hline
    
    \multicolumn{1}{|p{3.5cm}|}{\nohyphens{Usabilidad}} & \multicolumn{1}{|c|}{X} & \multicolumn{1}{|p{10cm}|}{Existe una contribución al reconocimiento de la adecuación, ya que un usuario, al poder visualizar el historial de desarrollo completo en Git, podrá entender si el software, a nivel técnico, es adecuado para sus necesidades.} \\ \hline
    
    \multicolumn{1}{|p{3.5cm}|}{\nohyphens{Fiabilidad}} & \multicolumn{1}{|c|}{X} & \multicolumn{1}{|p{10cm}|}{Gracias al sistema de despliegue que detalla, este requerimiento no funcional contribuye a la madurez, disponibilidad y tolerancia a fallos del sistema. Al contar con un sistema de despliegue y versionado del proyecto, los desarrolladores pueden trabajar en el y hacer llegar actualizaciones muchísimo más rápido al entorno de producción, y por ende a los usuarios finales del software.} \\ \hline
    
    \multicolumn{1}{|p{3.5cm}|}{\nohyphens{Seguridad}} & \multicolumn{1}{|c|}{X} & \multicolumn{1}{|p{10cm}|}{Contribuye al no repudio ya que, gracias a que el control de versiones lleva registro de todo lo que se modifica en la base de código, será posible rastrear malas decisiones o errores con facilidad.} \\ \hline
    
    \multicolumn{1}{|p{3.5cm}|}{\nohyphens{Mantenibilidad}} & \multicolumn{1}{|c|}{X} & \multicolumn{1}{|p{10cm}|}{Este requerimiento contribuye a la modularidad, reusabilidad analizabilidad y a la capacidad del software para ser modificado y probado. Todo esto gracias a que los cambios realizados en el software son documentados a través del control de versiones.} \\ \hline
    
    \multicolumn{1}{|p{3.5cm}|}{\nohyphens{Portabilidad}} & \multicolumn{1}{|c|}{X} & \multicolumn{1}{|p{10cm}|}{Contribuye a la adaptabilidad del software, y a la capacidad del mismo para ser instalado. Como todo el software es público y se encuentra en Git, analizarlo para adaptarlo resulta muy sencillo. Por otra parte, el sistema de despliegue facilita enormemente el instalar la aplicación en un entorno determinado.} \\
    \hline
  \end{tabular}
  
  \captionof{table}{Tabla de requerimiento no funcional de control de versiones}\label{table:rnf:vcs}
\end{center}

\section{Interfaces Internas de Salida}

\begin{center}
	\begin{tabular}{ | c | p{3.5cm} | p{10cm} |}
		\hline
		\textbf{Id} & {\textbf{Nombre}} & {\textbf{Detalle de Datos}} \\ \hline
		{\textbf{IN\_01}} & Car & name, speed, accel, weight, multiplier, folder\_name, category, stock, season \\ \hline
		{\textbf{IN\_02}} & Track &  name, short\_name, difficulty, lenght, folder\_name, stock, season \\ \hline
		{\textbf{IN\_03}} & Season & name, start\_date, end\_date, current, racer\_result\_entries \\ \hline
		{\textbf{IN\_04}} & Ranking & number, racer\_result\_entries, season \\ \hline
		{\textbf{IN\_05}} & Session & number, host, version, physics, protocol, pickups, date, teams, category, session\_log\_data, ranking, races, racer\_result\_entries \\ \hline
		{\textbf{IN\_06}} & Race & track\_name, laps, racers\_count, racer\_entries \\ \hline
		{\textbf{IN\_07}} & RacerEntry & car\_name, position, username, time, best\_lap, finished, cheating, laps, racers\_count\\ \hline
		{\textbf{IN\_08}} & RacerResultEntry & username, country, session\_count, race\_count, positions\_sum, average\_position, obtained\_points, official\_score, participation\_multiplier, team \\ \hline
		{\textbf{IN\_09}} & User & username, encrypted\_password, reset\_password\_token, reset\_password\_sent\_at, remember\_created\_at, sign\_in\_count, current\_sign\_in\_at, last\_sign\_in\_at, current\_sign\_in\_ip, last\_sign\_in\_ip, confirmation\_token, confirmation\_token, confirmed\_at, confirmation\_sent\_at, unconfirmed\_email, failed\_attempts, unlock\_token, locked\_at, admin, mod, organizar, locale, country, profile, stats \\ \hline
	\end{tabular}
  
    \captionof{table}{Tabla de interfaces internas de salida}\label{table:interfaces:out}
\end{center}

\newpage

\section{Interfaces Externas de Salida}

\begin{center}
	\begin{tabular}{ | c | p{2cm} | p{7.5cm} | p{4cm} |}
		\hline
		{\textbf{Id}} & 	{\textbf{Nombre}} & {\textbf{Detalle de Datos}} & {\textbf{Medio de Salida}} \\
		\hline
		{\textbf{OUT\_01}} & Car & name, speed, accel, weight, multiplier, folder\_name, category, stock, season & Pantalla \\ \hline
		{\textbf{OUT\_02}} & Track & name, short\_name, difficulty, lenght, folder\_name, stock, season & Pantalla \\ \hline
		{\textbf{OUT\_03}} & Season & name, start\_date, end\_date, current, racer\_result\_entries & Pantalla \\ \hline
		{\textbf{OUT\_04}} & Ranking & number, racer\_result\_entries, season & Pantalla \\ \hline
		{\textbf{OUT\_05}} & Session & number, host, version, physics, protocol, pickups, date, teams, category, session\_log\_data, ranking, races, racer\_result\_entries & Pantalla, Archivo CSV. \\ \hline
		{\textbf{OUT\_06}} & User & username, admin, mod, organizer, locale, country, profile, stats & Pantalla \\ \hline
	\end{tabular}
  
  \captionof{table}{Tabla de interfaces externas internas de salida}\label{table:interfaces:out2}
\end{center}
