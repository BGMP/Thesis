Dentro de las comunidades del videojuego de carreras Re-Volt, 1999, Re-Volt America se ha quedado atrás en lo que refiere a su sistema interno de cálculo de resultados y estadísticas por jugador. Esta problemática genera el surgimiento de este proyecto, el cual pretende modernizar y mejorar la gestión interna de Re-Volt America, así como también busca enriquecer la experiencia de usuario como nunca antes se ha visto en la escena de las comunidades de Re-Volt.

El presente informe trata de una aplicación que busca centralizar toda la información relacionada a las sesiones multijugador del videojuego Re-Volt celebradas por la comunidad de Re-Volt America, logrando así mejorar la experiencia de usuario para los administradores encargados de mantener los registros de resultados y tablas de puntuación, además de servir como un cambio revolucionario para todos aquellos que buscan conseguir una experiencia competitiva dentro del videojuego. 
