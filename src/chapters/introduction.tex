El presente informe contiene las especificaciones técnicas correspondientes al desarrollo del proyecto de titulación de la carrera de Ingeniería de ejecución Informática titulado ''Desarrollo de una aplicación web para la comunidad de Re-Volt America''.

El documento se organiza en varios capítulos. Primeramente, el capítulo 1 trata del estudio del problema, que es donde se habla de la historia del videojuego Re-Volt y sus comunidades, así como también del contexto del problema al que se enfrenta Re-Volt America hoy en día. Luego, en el capítulo 2 se definen los objetivos generales y específicos del proyecto a desarrollar, además de la metodología de desarrollo elegida.

Más adelante, en el capítulo 3 de factibilidad, podrá encontrarse una descomposición de la factibilidad del proyecto en tres pilares fundamentales: factibilidad técnica, factibilidad operativa, y factibilidad económica. Este capítulo termina entregando una conclusión de factibilidad.

El capítulo 4 contiene todo lo que tiene que ver con la arquitectura de Re-Volt America a nivel técnico. Desde su lógica de negocio hasta su sistema de despliegue en la actualidad.

En el capítulo 5, se definirán los límites del software, la caracterización de sus usuarios, los objetivos tanto generales como específicos, los requerimientos del software y las interfaces.

En el capítulo 6 se realiza un análisis funcional a todo el proyecto ya desarrollado, evaluando sus actores, casos de uso, modelos de datos, esquema de base de datos, diseño de interfaz, arquitectura y estructura del código.

En el capítulo 7, se habla del plan de capacitación, implantación y puesta en marcha del proyecto.

Finalmente, el capítulo 8 entrega una conclusión del proyecto, y el capítulo 9 adjunta todos los anexos relevantes.