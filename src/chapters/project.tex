\chapter{Proyecto}

\section{Objetivo General del Proyecto}
Desarrollar una aplicación web para Re-Volt America, la comunidad del continente americano formada alrededor del videojuego de carreras Re-Volt, 1999.

\section{Objetivos Específicos del Proyecto}
\begin{enumerate}
	\item Elaborar una propuesta que consiga atender las necesidades y problemas de los usuarios de Re-Volt America, en relación con el almacenamiento y visualización de resultados de partidas online, además de proveer visibilidad a la comunidad en general.
	\item Diseñar la solución de software de procesamiento de datos de sesiones multijugador de Re-Volt, creando interfaces que les permitan a los jugadores visualizar los resultados oficiales de las sesiones de carreras en línea, además de estadísticas personales.
	\item Implementar la aplicación web, la cual permitirá a los organizadores de sesiones de carreras en línea subir y publicar los resultados de dichas carreras, además de realizar el lazo entre jugadores y cuentas de usuario de esta. El software implementado permitirá, a su vez, procesar dicha información subida a la web, y así mostrar a los usuarios finales una vista clara de sus resultados en carreras y sus estadísticas personales.
\end{enumerate}

\section{Metodología de Desarrollo}

INSERTAR TABLA DE RIESGO

Luego de analizar a fondo el proyecto, se concluyó que la experiencia en el área y la complejidad de la problemática son altas, pero el tamaño del problema es pequeño, lo cual representa un bajo riesgo en lo que respecta al problema. Por otra parte, en relación con el software se cuenta con una alta experiencia en términos técnicos. Además, el software en si tiene una complejidad y tamaño pequeños, por lo que el riesgo es realmente bajo en lo que a este respecta. En conclusión, el riesgo ponderado entre el problema y el software a desarrollar es muy bajo.

Según la evaluación del proyecto, se concluyó que este tiene un riesgo total asociado muy bajo, por lo que se contaría con la libertad de utilizar cualquier tipo de metodología de desarrollo para llevarlo a cabo. Como contamos con esta libertad de elección, se buscará sacar provecho de ella, y se ha seleccionado una metodología de desarrollo iterativa, a través de la cual se desarrollará en ciclos. Estos ciclos permitirán al software evolucionar a medida que se recibe retroalimentación por parte de los usuarios, corrigiendo errores y mejorando detalles a medida que se progresa en el desarrollo de la aplicación

Según el contexto en el que se desarrollará esta aplicación, los usuarios de la comunidad tendrán una importante incidencia en el testeo y uso diario de la misma, lo cual implica que estarán presentes durante el proceso de implementación de varias de las funcionalidades que se pretenden lograr con esta propuesta. Se sabe también con antelación que los usuarios estarán dispuestos a ayudar con las pruebas y el testeo de la aplicación. Esta opción fue seleccionada ya que:

\begin{enumerate}
	\item Permite más flexibilidad en cuanto a lo que se planea desarrollar como aplicación web, debido a que a medida que avance el proyecto es muy probable que surjan cambios o nuevas ideas a partir de la retroalimentación recibida de la comunidad.
	\item Debido a la naturaleza del proyecto, utilizar una metodología iterativa es muy beneficioso ya que ésta permitirá una constante supervisión de los cambios realizados al software, lo que finalmente se traducirá en menos errores, y un producto final que se ajuste y esté a la altura de las necesidades de la comunidad.
	\item Con una metodología iterativa será posible contar con prototipos que la comunidad pueda comenzar a utilizar, y a su vez adaptarse a las interfaces y funcionalidades de la aplicación mientras esta se encuentra en desarrollo. Esto es muy importante ya que las sesiones multijugador suelen organizarse a diario, por lo que mientras antes se cuente con una solución funcional, antes podrá la comunidad comenzar a llevar un registro histórico de las partidas online que organiza.
\end{enumerate}

\section{Técnicas y Notaciones}
\begin{itemize}
	\item Diagrama de Casos de Usos.
	\item BPMN para modelar el proceso de negocio actual.
	\item Carta Gantt para la planificación inicial del proyecto.
\end{itemize}

\section{Estándares de Documentación}
\begin{itemize}
	\item Adaptación Basada en IEEE Software Test Documentation Std 829-1998.
	\item Adaptación Basada en IEEE Software Requirements Specifications Std 830-1998.
\end{itemize}

\section{Software, Frameworks y  Lenguajes Utilizados}
\label{project:software}
A continuación se lista el software, frameworks y lenguajes de programación, marcado y estilos utilizados para la realización de este proyecto.

Para efectos del siguiente listado, los nombres de las herramientas, frameworks y lenguajes se han redactado en negrita, seguidos de paréntesis en itálica que contienen el número de la versión asociada a cada ítem.

\begin{itemize}
	\item[] \textbf{Lenguajes}
	\begin{itemize}
		\item \textbf{Ruby} \textit{(3.2.2)}: Lenguaje de programación de alto nivel.
		\item \textbf{HAML} \textit{(6.2.3)}: Lenguaje de marcado para la abstracción de HTML.
		\item \textbf{Sass} \textit{(6.0)}: Lenguaje de extensión para CSS.
	\end{itemize}
\end{itemize}

\begin{itemize}
	\item[] \textbf{Software}
	\begin{itemize}
		\item \textbf{MongoDB} \textit{(7.0.3)}: Base de datos orientada a documentos JSON.
		\item \textbf{Redis} \textit{(7.0.12)}: Almacenamiento en memoria, utilizado para el caché de datos.
		\item \textbf{RubyMine} \textit{(2023.2.2)}: Entorno de desarrollo integrado especializado para el trabajo con aplicaciones en Ruby, específicamente para Ruby on Rails.
		\item \textbf{MongoDB Compass} \textit{(1.39.0)}: Visor para bases de datos de MongoDB.
		\item \textbf{RedisInsight} \textit{(2.30.0)}: Visor para el almacenamiento del caché en Redis.
		\item \textbf{Docker Desktop} \textit{(4.21.0)}: Visor y gestor de contenedores de Docker, en formato de aplicación de escritorio multiplataforma.
		\item \textbf{NodeJS} \textit{(16.13.0)}: Entorno de servidor multiplataforma utilizado para la conversión de archivos en runtime.
		\item \textbf{Yarn} \textit{(1.22.21)}: Gestor de paquetes para JavaScript.
		\item \textbf{Docker} \textit{(24.0.2)}: Tecnología que permite crear y utilizar contenedores. Para efectos de este proyecto, es utilizado con el fin de probar el software desarrollado en distribuciones de Linux determinadas.
		\item \textbf{Termius} \textit{(8.7.2)}: Cliente SSH.
		\item \textbf{Git/Git Bash} \textit{(2.34.1)}: Sistema de control de versiones.
		\item \textbf{Ubuntu LTS} \textit{(18.04.6)}: Subsistema de Linux para Windows.
	\end{itemize}
\end{itemize}

\begin{itemize}
	\item[] \textbf{Frameworks}
	\begin{itemize}
		\item \textbf{Ruby on Rails} \textit{(7.1)}: Framework para desarrollo de aplicaciones web fullstack.
		\item \textbf{Bootstrap} \textit{(4.4.1)}: Framework para la creación, manejo de elementos visuales y la responsividad en aplicaciones web.
	\end{itemize}
\end{itemize}
