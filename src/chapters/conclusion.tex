\chapter{Conclusión del Proyecto}
Para finalizar este informe, a continuación, se plantean una serie de conclusiones a las que se llegó luego de llevar a cabo el desarrollo e implementación del software:

\begin{itemize}
	\item En relación con poder haber elaborado una propuesta que consiguiera atender las necesidades y problemas de los usuarios de Re-Volt America en relación con el almacenamiento y visualización de resultados de partidas online, podemos decir que, en conclusión, se logró cumplir dicho objetivo gracias al software desarrollado.
	\item De acuerdo con el objetivo que hablaba de diseñar una solución de software de procesamiento de datos de sesiones multijugador oficiales de las sesiones de carreras en línea, además de estadísticas personales, podemos decir que es un objetivo que fue cumplido parcialmente, ya que aún no se han implementado las estadísticas por jugador.
	\item En relación con el objetivo de implementar una aplicación web que permitiera a los organizadores de sesiones de carreras en línea subir y publica los resultados de dichas carreras podemos decir que, en conclusión, dicho objetivo fue cumplido con éxito, ya que el software, incluso en su estado actual, ya puede realizar la importación de Session Logs, el cálculo y procesamiento de resultados en el formato de RVA, y finalmente mostrar dichos resultados por pantalla a quien los solicite desde la web.
	\item El proyecto ha permitido demostrar, incluso en su estado actual, las competencias que se esperan de un ingeniero de ejecución en computación e informática, ya que he aplicado la identificación de necesidades, análisis y el diseño de soluciones informáticas para Re-Volt America, logrando desarrollar una solución que le permite a la comunidad tener un mejor manejo de sus proceso internos, registros de datos más fiables y una experiencia de usuario muchísimo mejor que con la que contaban al trabajar con su sistema original. 
	\item Bajo mi percepción, el proyecto fue realmente enriquecedor desde el punto de vista del desarrollo de software puesto que, dentro del área que fue comprendido pude aplicar diversas tecnologías, técnicas de diseño de software, despliegue de aplicaciones y aplicación de estándares de calidad, todo en un mismo contexto que cierra de manera redonda el ciclo de desarrollo que se espera pueda ser alcanzado por un ingeniero de software.
\end{itemize}
