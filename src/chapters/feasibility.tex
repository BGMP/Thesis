\chapter{Factibilidad}

\section{Factibilidad Técnica}
Se realizó un estudio previo de factibilidad técnica, el cual arrojó que las tecnologías que se utilizarán para desarrollar el software en cuestión son capaces de resolver el problema.
Descomponiendo los aspectos técnicos, podemos decir que está más que comprobado que Ruby on Rails ha sido utilizado para conseguir resultados similares con proyectos del mismo tipo, por lo que es completamente factible utilizarlo en el contexto de este proyecto.
Las tecnologías utilizadas y la manera en que se montará el proyecto permitirá también que este sea perfectamente escalable y mejorable.
Es por esto que es factible el desarrollo de este proyecto.

\section{Factibilidad Operativa}
La factibilidad operativa es fácilmente demostrada al comparar la solución propuesta con el sistema que pretende reemplazar. Vale decir que actualmente quienes hacen uso del sistema de hojas de cálculo maestras y la aplicación de escritorio para el procesamiento de sesiones actualmente en RVA verían su trabajo facilitado en todos los sentidos al contar con una plataforma que se encargue de llevar la cuenta de todo, procesar los resultados y mantener las tablas de resultados y rankings de temporadas de manera automática.

\section{Factibilidad Económica}
Agregar factibilidad económica...

\section{Conclusión de Factibilidad}
Gracias al análisis realizado en los puntos anteriores, se puede concluir que el proyecto es completamente factible. Podrá ser desarrollado de manera efectiva y esto asegura una mejora sustancial del proceso actual de procesamiento de datos de las sesiones de RVA y manejo de resultados en general.
