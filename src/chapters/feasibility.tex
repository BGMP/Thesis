\chapter{Factibilidad}

\section{Factibilidad Técnica}
\subsection{Conocimientos de los Usuarios}
Para el correcto funcionamiento de la aplicación propuesta, es de esperar que los distintos miembros del staff de RVA, quienes serán los principales usuarios del software, deban tener determinados conocimientos para poder operarla con éxito.

Gracias a que, naturalmente, el staff de RVA lleva un largo tiempo manejando la comunidad y los resultados de las sesiones organizadas, no hace falta mayor capacitación técnica en cuanto al funcionamiento del cálculo de puntos, multiplicadores de autos, y demás características específicas del sistema de RVA. Para efectos de factibilidad técnica, el staff ya cuenta con los conocimientos necesarios para poder migrar a un sistema que sólo busca mejorar y facilitar los procesos actuales.

En cuanto a la aplicación propuesta en este proyecto y su funcionamiento operacional, el staff será habituado al nuevo sistema mediante un video de inducción a la nueva plataforma web. También se asignará un periodo de prueba para que puedan utilizar la página ellos mismos, y así puedan adaptarse fácilmente.

\subsection{Disponibilidad Profesional}
Para el desarrollo de la aplicación propuesta, se necesita del trabajo de un profesional en el área del desarrollo de software, el cual sea capaz de satisfacer las necesidades de RVA y cumplir los objetivos de desarrollo propuestos.

Para efectos de este proyecto, se cuenta tanto con el tiempo profesional, como también con los conocimientos técnicos requeridos.

Adicionalmente, se cuenta con el equipo físico para poder desarrollar la aplicación, como puede ser un computador, acceso a internet y demás software para desarrollo. A continuación, se presentan tablas de especificación de los equipos físicos y el software con el que se cuenta para desarrollar la aplicación.

% Equipos Físicos
\begin{center}
	\begin{tabular}{ | l | p{10cm} |}
		\hline
		\multicolumn{2}{|c|}{\textbf{Equipos Físicos}} \\
		\hline
		\multicolumn{1}{|c|}{\textbf{Nombre}} & \multicolumn{1}{|c|}{\textbf{Acceso}} \\
		\hline
		{\textbf{Windows PC}} & Equipo personal \\ \hline
		
		{\textbf{MacBook Pro M1}} & Equipo personal \\ \hline
	\end{tabular}
\end{center}

% Software
\begin{center}
	\begin{tabular}{ | l | p{10cm} |}
		\hline
		\multicolumn{2}{|c|}{\textbf{Software}} \\
		\hline
		\multicolumn{1}{|c|}{\textbf{Nombre}} & \multicolumn{1}{|c|}{\textbf{Acceso}} \\
		\hline
		
		{\textbf{Notepad++}} & Software libre \\ \hline
		
		{\textbf{MongoDB Compass}} & Software libre \\ \hline
		
		{\textbf{RedisInsight}} & Software libre \\ \hline
		
		{\textbf{Git/Git Bash}} & Software libre \\ \hline
		
		{\textbf{Ubuntu LTS}} & Software libre \\ \hline
		
		{\textbf{RubyMine}} & Licencia de estudiante \\ \hline
		
		{\textbf{Termius}} & Licencia de estudiante \\ \hline
		
		{\textbf{Microsoft Excel}} & Licencia de estudiante \\ \hline
	\end{tabular}
\end{center}

\subsection{Despliegue y Servidor}
Para realizar el despliegue de la aplicación a desarrollar, se ha escogido un servidor de tipo VPS (Servidor Virtual Privado). Las especificaciones técnicas de dicho servidor son las siguientes:

% Servidor
\begin{center}
	\begin{tabular}{ | l | p{10cm} |}
		\hline
		\multicolumn{2}{|c|}{\textbf{VPS}} \\
		\hline
		\multicolumn{1}{|c|}{\textbf{Característica}} & \multicolumn{1}{|c|}{\textbf{Detalle}} \\
		\hline
		
		{\textbf{Proveedor}} & DigitalOcean \\ \hline
		
		{\textbf{Región}} & Nueva York \\ \hline
		
		{\textbf{Sistema Operativo}} & Ubuntu 22.04 (LTS) x64 \\ \hline
		
		{\textbf{Tipo de CPU}} & Intel Regular \\ \hline
		
		{\textbf{Número de vCPUs}} & 1 CPU\\ \hline
		
		{\textbf{Memoria}} & 2 GB \\ \hline
		
		{\textbf{Almacenamiento (SSD)}} & 50 GB \\ \hline
		
		{\textbf{Transferencia}} & 2 TB \\ \hline
	\end{tabular}
\end{center}

El software instalado en este servidor para el despliegue de la aplicación es el mismo especificado en la \autoref{project:software}.

\section{Factibilidad Operativa}
En cuanto a la factibilidad operativa de este proyecto, se sabe que el staff de Re-Volt America tiene gran disposición al cambio, ya que el sistema antiguo no solamente les demanda demasiado tiempo y esfuerzo, sino que, además, resulta poco preciso y muy propenso a errores en su uso diario debido a la gran cantidad de pasos manuales que este conlleva.

La factibilidad operativa es fácilmente demostrada al comparar la solución propuesta con el sistema que esta pretende reemplazar. Vale decir que, actualmente, quienes hacen uso del sistema de hojas de cálculo maestras en Excel y la aplicación de RVA-Points para el procesamiento de las sesiones, verían su trabajo facilitado en todos los sentidos al contar con una plataforma que se encargue de llevar la cuenta de todo, procesar los resultados y mantener los rankings y temporadas al día de manera automática y confiable.

\section{Factibilidad Económica}
A continuación se presenta un detalle de tablas de costos y el flujo de caja asociado al proyecto. 

\subsection{Tablas de Costos}
\label{feasibility:costs}
En esta sección se presentan dos tablas de costos relacionadas al proyecto. La primera tabla contiene todo el software utilizado para el desarrollo de este proyecto, mientras que la segunda tabla contempla los costos inherentes de RVA, combinando aquellos costos adquiridos que vienen de antes del despliegue a producción del software que se ha desarrollado.

% Equipos Físicos
\begin{center}
	\begin{tabular}{ | l | p{5cm} | p{5cm}|}
		\hline
		\multicolumn{3}{|c|}{\textbf{Software}} \\
		\hline
		\multicolumn{1}{|c|}{\textbf{Nombre}} & \multicolumn{1}{|c|}{\textbf{Acceso}} & \multicolumn{1}{|c|}{\textbf{Precio (anual)}} \\
		\hline
		{\textbf{Notepad++}} & Software libre & \multicolumn{1}{|r|}{\$0} \\ \hline

		{\textbf{MongoDB Compass}} & Software libre & \multicolumn{1}{|r|}{\$0} \\ \hline
		
		{\textbf{RedisInsight}} & Software libre & \multicolumn{1}{|r|}{\$0} \\ \hline
		
		{\textbf{Git/Git Bash}} & Software libre & \multicolumn{1}{|r|}{\$0} \\ \hline
		
		{\textbf{Ubuntu LTS}} & Software libre & \multicolumn{1}{|r|}{\$0} \\ \hline
		
		{\textbf{RubyMine}} & Licencia de estudiante & \multicolumn{1}{|r|}{\$0} \\ \hline
		
		{\textbf{Termius}} & Licencia de estudiante & \multicolumn{1}{|r|}{\$0} \\ \hline
		
		{\textbf{Microsoft Excel}} & Licencia de estudiante & \multicolumn{1}{|r|}{\$0} \\ \hline
	\end{tabular}
\end{center}

% Costos de Producción
\begin{center}
	\begin{tabular}{ | l | p{5cm} | p{5cm}|}
		\hline
		\multicolumn{3}{|c|}{\textbf{Costos de Producción}} \\
		\hline
		\multicolumn{1}{|c|}{\textbf{Nombre}} & \multicolumn{1}{|c|}{\textbf{Proveedor}} & \multicolumn{1}{|c|}{\textbf{Precio (anual)}} \\
		\hline
		{\textbf{VPS}} & DigitalOcean & \multicolumn{1}{|r|}{\$125.000} \\ \hline
		
		{\textbf{Mailer}} & Postmark & \multicolumn{1}{|r|}{\$156.000} \\ \hline
		
		{\textbf{Dominio (rva.lat)}} & Namecheap & \multicolumn{1}{|r|}{\$22.600} \\ \hline
		
		{\textbf{Git/Git Bash}} & Software libre & \multicolumn{1}{|r|}{\$0} \\ \hline
		
		{\textbf{Ubuntu LTS}} & Software libre & \multicolumn{1}{|r|}{\$0} \\ \hline
		
		{\textbf{RubyMine}} & Licencia de estudiante & \multicolumn{1}{|r|}{\$0} \\ \hline
		
		{\textbf{Termius}} & Licencia de estudiante & \multicolumn{1}{|r|}{\$0} \\ \hline
		
		{\textbf{Microsoft Excel}} & Licencia de estudiante & \multicolumn{1}{|r|}{\$0} \\ \hline
	\end{tabular}
\end{center}

Cabe destacar que, dentro de los costos de producción, tanto el precio anual del VPS como el del dominio de RVA son costos adquiridos que han sido mantenidos desde antes de la puesta en marcha de este proyecto. Su mención en la tabla anterior es puramente una formalidad.

\subsection{Flujos de Caja}
Para la redacción del flujo de caja y la confección de sus tablas asociadas, se debe tener en cuenta que existen contextos económicos internos diferentes dentro de la evolución de RVA como proyecto.

En segundo lugar, se hablará de una proyección de lo que costará, en términos económicos, mantener el software desarrollado para este proyecto a futuro desde su finalización.

La valorización del tiempo de trabajo será determinada, en ambos casos, a partir de promedios y estimaciones que pueden encontrarse hoy en día en el mercado del desarrollo de software. Las referencias serán mencionadas oportunamente.

\subsubsection{Desarrollo del Proyecto}
Para poder realizar este flujo de caja, se han hecho algunas valorizaciones esenciales. En primer lugar, cuando se hace referencia a la inversión inicial, se habla de todo aquello que ha sido valorizado desde los inicios de la modernización de RVA (2021), hasta el comienzo del presente proyecto. Ejemplos de ello pueden ser el dominio de ''rva.lat'', o el tiempo profesional invertido en el desarrollo previo al inicio de este proyecto, ya que ambas son cosas que preceden a la puesta en marcha del proyecto como tal.

La tasa mínima aceptable de rendimiento (TMAR) para este proyecto ha sido calculada a partir de valores promedios reportados por el Banco Central de Chile (inflación), y por el Standish Group International en 2011 (CHAOS Manifesto) (prima de riesgo):

% Costos de Producción
\begin{center}
	\begin{tabular}{ | p{7cm} | p{5cm}|}
		\hline
		{\textbf{Inflación Promedio}} & 4.3\%  \\ \hline
		{\textbf{Tasa Prima de Riesgo}} & 7\% \\ \hline
	\end{tabular}
\end{center}


\[
\mathlarger{
	r = 0.043+0.21\cdot(0.043\cdot0.21) = 26.2\%
}
\]

A partir de estos datos, es posible realizar un cálculo del valor actual de costos (VAC) del proyecto. En este caso, todos los costos asociados figurarán como un ahorro para la comunidad de RVA, ya que debemos considerar que todo lo realizado vendría representar dinero que, potencialmente, la comunidad hubiese tenido que gastar en otro profesional y recursos de no haberse desarrollado el software en cuestión.

A continuación, se desglosa el cálculo del VAC en un periodo de 6 meses, con una inversión inicial calculada a partir de la tabla de costos de la \autoref{feasibility:costs}, extrapolada a 2 años.

La valorización del tiempo empleado en desarrollo de software a lo largo del proyecto está basada en el sueldo promedio de desarrollador en Chile según talent.com (\$1.262.000 mensual; \$7.769 por hora), ajustado a las horas de trabajo efectivas empleadas en el proyecto, las cuales fueron 4 horas de trabajo efectivo durante los 7 días de la semana por mes de desarrollo.

\[
\mathlarger{
	Desarrollo = \$7.769 \cdot \left(4h\cdot7d\cdot4w\right) = \$870.128
}
\]


% VAC
\begin{center}
	\begin{tabular}{ | l | l | l | l | l | l | l | l |}
		\hline
		\multicolumn{8}{|c|}{\textbf{VAC}} \\
		\hline
		 & \multicolumn{1}{|c|}{\textbf{0}} & \multicolumn{1}{|c|}{\textbf{1}} & \multicolumn{1}{|c|}{\textbf{2}} & \multicolumn{1}{|c|}{\textbf{3}} & \multicolumn{1}{|c|}{\textbf{4}} & \multicolumn{1}{|c|}{\textbf{5}} & \multicolumn{1}{|c|}{\textbf{6}} \\
		\hline
		{\textbf{Desarrollo}} &  & \multicolumn{1}{|r|}{\$870.128} & \multicolumn{1}{|r|}{\$870.128} & \multicolumn{1}{|r|}{\$870.128} & \multicolumn{1}{|r|}{\$870.128} & \multicolumn{1}{|r|}{\$870.128} & \multicolumn{1}{|r|}{\$870.128} \\ \hline
		
		{\textbf{Internet}} &  & \multicolumn{1}{|r|}{\$15.000} & \multicolumn{1}{|r|}{\$15.000} & \multicolumn{1}{|r|}{\$15.000} & \multicolumn{1}{|r|}{\$15.000} & \multicolumn{1}{|r|}{\$15.000} & \multicolumn{1}{|r|}{\$15.000} \\ \hline
		
		{\textbf{Electricidad}} &  & \multicolumn{1}{|r|}{\$36.600} & \multicolumn{1}{|r|}{\$36.600} & \multicolumn{1}{|r|}{\$36.600} & \multicolumn{1}{|r|}{\$36.600} & \multicolumn{1}{|r|}{\$36.600} & \multicolumn{1}{|r|}{\$36.600} \\ \hline
		
		{\textbf{Hosting}} &  & \multicolumn{1}{|r|}{\$10.526} & \multicolumn{1}{|r|}{\$10.526} & \multicolumn{1}{|r|}{\$10.526} & \multicolumn{1}{|r|}{\$10.526} & \multicolumn{1}{|r|}{\$10.526} & \multicolumn{1}{|r|}{\$10.526} \\ \hline
		
		{\textbf{Flujo}} &  & \multicolumn{1}{|r|}{\$932.254} & \multicolumn{1}{|r|}{\$932.254} & \multicolumn{1}{|r|}{\$932.254} & \multicolumn{1}{|r|}{\$932.254} & \multicolumn{1}{|r|}{\$932.254} & \multicolumn{1}{|r|}{\$932.254} \\ \hline
		& \multicolumn{1}{|r|}{\$606.000} & & & & & & \\ \hline
		\textbf{Flujo Total} & \multicolumn{1}{|r|}{\$5.593.524} & & & & & & \\ \hline
	\end{tabular}
\end{center}

Teniendo en cuenta la tabla TABLENUM, podemos calcular el VAC integrando en su fórmula los valores obtenidos:

\[
\mathlarger{
	{VAC} = I_0 + \sum\limits_{t=i}^{n}\frac{C_t}{\left(1+r^t\right)} = \$2.677.422
}
\]


https://www.bcentral.cl/web/banco-central/contenido/-/detalle/informe-de-politica-monetaria-septiembre-2023 (INFLACIÓN)

https://www.immagic.com/eLibrary/ARCHIVES/GENERAL/GENREF/ChaosManifest\_2011.pdf (TASA DE RIESGO)

\section{Conclusión de Factibilidad}
Para concluir, podemos decir que en cuanto a los distintos tipos de factibilidad evaluados se cuenta con usuarios técnicamente capaces y con la disponibilidad profesional necesaria para el desarrollo del proyecto. Por el lado operativo, existe disposición al cambio por parte de los administradores y el staff en general. Económicamente, a partir de los costos y beneficios recopilados, se ha obtenido que existe un valor económico sustancial asociado al proyecto.

Gracias al análisis realizado en los puntos anteriores, se puede concluir que el proyecto es factible.
