\documentclass[12pt]{report}
\usepackage[hidelinks]{hyperref}
\usepackage{array}
\usepackage[spanish]{babel}
\usepackage{color}
\usepackage{colortbl}
\usepackage[utf8]{inputenc}
\usepackage{geometry}
\usepackage{graphicx}
\usepackage{tabularray}
\graphicspath{{img/}}

\newgeometry{
	right=3cm,
	left=3cm,
	top=2.5cm,
	bottom=2.5cm
}

\hypersetup{
	colorlinks,
	citecolor=black,
	filecolor=black,
	linkcolor=black,
	urlcolor=black
}

\definecolor{Black}{rgb}{0,0,0}

% Índice
\addto\captionsspanish{
	\renewcommand{\contentsname}%
	{Índice}%
}

\begin{titlepage}

\title{
	\begin{figure}
		\centering
			\includegraphics[width=2cm, height=3cm]{img/logo.png}\\
			{Universidad del Bío-Bío\\
			Facultad de Ciencias Empresariales\\
			Depto. de Sistemas de Información}
	\end{figure}
	{Desarrollo de una aplicación web para la comunidad de Re-Volt America}\\
	{\large Proyecto de título para optar al título de Ingeniero de Ejecución en Computación e Informática}
}
\author{José Benavente}
\date{Viernes 6 de octubre, 2023}

\end{titlepage}

\begin{document}

\maketitle

\tableofcontents

\chapter*{Abstracto}
\addcontentsline{toc}{section}{Abstracto}
\input{chapters/abstract.tex}

\chapter*{Dedicatoria}
\addcontentsline{toc}{section}{Dedicatoria}
El presente proyecto está dedicado a toda la comunidad de Re-Volt, en especial a quienes forman parte del grupo de jugadores de Re-Volt America.

Sin lugar a dudas, durante los últimos años, este juego pasó de ser un simple pasatiempo a convertirse en algo muy importante para mí a nivel personal. Muchas de las personas que he conocido a través de Re-Volt se han convertido, ya a estas alturas, en buenos amigos a quienes valoro muchísimo. Todos han sabido siempre apreciar mi trabajo, y por eso les estoy infinitamente agradecido. Sin ustedes, nada de lo que amo hacer tendría trascendencia alguna. De todo corazón, muchas gracias. Es por todo lo anterior que no podría dedicar este trabajo a nadie si no a ustedes.


\chapter*{Agradecimientos}
\addcontentsline{toc}{section}{Agradecimientos}
En primer lugar, se extiende un agradecimiento formal a los siguientes desarrolladores quienes, de una forma u otra, han contribuido a la base de código del presente proyecto:
\begin{itemize}
	\item Marco Roth. Por ayudar con la migración del proyecto a ''esbuild'', y por resolver problemas con HAML.
	\item Nicolás Duque. Por probar la instalación del proyecto en plataformas de Linux, testear el proyecto en su fase beta, y por ayudar con la implementación de traducciones en la web.
\end{itemize}

También se les extiende un agradecimiento a todos quienes han hecho una donación voluntaria, sin importar el monto que fuese, al desarrollador principal de este proyecto a través de GitHub Sponsors durante el desarrollo de esta tesis:
\begin{itemize}
	\item Vicente Aguilera.
	\item Gabriel Carnielli.
	\item Dario Chaile.
	\item Benjamín Contreras.
	\item Benjamín Ferrada.
	\item Josafat jiménez.
	\item  Mateusz Kobylański.
	\item Jorge Matamala.
	\item Benjamín Mosso.
	\item Leandro Rodríguez.
	\item Juan Pablo Rosas.
\end{itemize}


\chapter*{Resumen}
\addcontentsline{toc}{section}{Resumen}
VENDER REVOLT AMERICA:

RV AMERICA HA QUEDADO ATRAS EN CUANTO A SU SOPORTE Y GRACIAS A ESTE PROYECTO CRECERÁ Y SE VOLVERÁ MÁS MASIVO ETC...


El presente informe trata de una aplicación que busca centralizar toda la información relacionada a las sesiones multijugador del videojuego Re-Volt celebradas por la comunidad de Re-Volt America, la cual busca mejorar la experiencia de usuario para los administradores encargados de mantener los registros de resultados y tablas de puntuación de Re-Volt America actualizadas, además de servir como un cambio revolucionario para todos aquellos que buscan conseguir una experiencia competitiva dentro del videojuego. 


\chapter*{Introducción}
\addcontentsline{toc}{section}{Introducción}
Dentro de las comunidades del videojuego de carreras Re-Volt, 1999, Re-Volt America se ha quedado atrás en lo que refiere a su sistema interno de cálculo de resultados y estadísticas por jugador. Esta problemática genera el surgimiento de este proyecto, el cual pretende modernizar y mejorar la gestión interna de Re-Volt America, así como también busca enriquecer la experiencia de usuario como nunca antes se ha visto en la escena de las comunidades de Re-Volt.

El presente informe trata de una aplicación que busca centralizar toda la información relacionada a las sesiones multijugador del videojuego Re-Volt celebradas por la comunidad de Re-Volt America, logrando así mejorar la experiencia de usuario para los administradores encargados de mantener los registros de resultados y tablas de puntuación, además de servir como un cambio revolucionario para todos aquellos que buscan conseguir una experiencia competitiva dentro del videojuego. 


% Estudio del Problema
\chapter{Estudio del Problema}

\section{Definiciones, Siglas y Abreviaciones}
A continuación, se definirán algunos conceptos relevantes en el contexto del videojuego Re-Volt y la comunidad de Re-Volt America:

\begin{itemize}
	\item Re-Volt: El videojuego Re-Volt, 1999 (https://en.wikipedia.org/wiki/Re-Volt).
	\item RV: Re-Volt.
	\item RVGL: Re-Volt: OpenGL. La reescritura del juego original que es usada por todos los jugadores hoy en día. (https://rvgl.org).
	\item Re-Volt I/O: Comunidad Europea de Re-Volt.
	\item RVA: Re-Volt America.
	\item Sesión: Evento de carreras online del videojuego Re-Volt, en donde dos o más personas compiten en dos o más carreras multijugador.
	\item Session Log: Archivo separado por comas que contiene un registro crudo de los resultados de las carreras jugadas en una sesión de RVGL.
\end{itemize}

\section{Historia de Re-Volt}
Re-Volt America, en su expresión más simple, es una comunidad de jugadores del videojuego Re-Volt, el cual fue lanzado originalmente en el año 1999 por Acclaim Studios en Londres. Re-Volt es un videojuego de carreras y simulador arcade de autos a control remoto, el cual explora una premisa en donde dichos autos compiten en carreras de radio control en ambientes como museos, supermercados, barcos, sitios de construcción, entre otros. Esto combinado con una mecánica de objetos que pueden ser recogidos por dichos autos para atacar a los competidores, obtener más velocidad, entre otras ventajas.

El arte original de la caratula del juego puede ser apreciado en la figura FIGNUM.

\includegraphics{img/re-volt.jpg}

El primero de septiembre del año 2004, Acclaim Studios se declara en banca rota, y cesa permanentemente todo el desarrollo y mantenimiento que en algún momento proveyó a Re-Volt y a su comunidad. Este suceso, a lo largo de los años, dio lugar a muchas comunidades segmentadas del juego en el internet de ese entonces. Con el tiempo, nuevos sitios y proyectos comenzaron a surgir, tales como el portal web de Re-Volt Race, una página de Re-Volt que se dedicaba a organizar partidas online y mantener tablas de resultados para los jugadores, o Re-Volt: OpenGL (RVGL), una re-escritura moderna del Re-Volt original que todos conocían, ahora disponible para plataformas modernas y otros sistemas operativos además de Windows, como Linux, MacOS, e incluso una versión para dispositivos Android.

Dentro de lo anteriormente enmarcado, aparece en el año 2015 la comunidad de Re-Volt I/O, cuyo logotipo se puede apreciar en la figura FIGNUM. Esta comunidad estaba formada por un grupo de jugadores de Re-Volt, principalmente europeos, quienes incursionaron por primera vez en intentar crear una plataforma estable para el videojuego y su comunidad de jugadores. Este sería un lugar en donde cualquiera que quisiera disfrutar del juego podría encontrar guías de ayuda, tutoriales, descargas y demás contenido para poder instalar y jugar Re-Volt en su computador o dispositivo móvil.

\includegraphics{img/io.png}

En sus inicios, Re-Volt I/O adoptó a RVGL como la distribución estándar de Re-Volt que ofrecería a sus jugadores, haciéndole ganar público y reconocimiento al proyecto publicando enlaces de descarga directos en su página web (re-volt.io), además de entregar soporte y mantener hilos de discusión relacionados con RVGL y sus actualizaciones en su foro oficial (forum.re-volt.io).

En adición a lo anterior, RVGL no era tan sólo una versión modernizada del Re-Volt original, sino que también traía consigo el aspecto más importante que tiene Re-Volt en la actualidad, y el cual mantiene unida y activa a su comunidad en general: el modo multijugador u online. Dicho modo no sólo permitía a los jugadores correr carreras en línea, sino que, además, extendía soporte para que miembros de la comunidad pudiesen diseñar sus propios autos y pistas de manera personalizada, agrandando así, de manera casi infinita, el repertorio de contenido descargable para Re-Volt.

Re-Volt I/O adoptó un sistema en donde su administración elige ciertos autos y pistas hechos por la comunidad cada ciertos meses. De esta forma, todos estos autos y pistas, elegidos a votación, terminan juntos en un paquete de contenido de extensión para RVGL, el cual Re-Volt I/O se encarga de distribuir para que sus usuarios lo descarguen y puedan jugar en línea. De manera habitual, tener este paquete de contenido es obligatorio para poder jugar en las sesiones multijugador organizadas por Re-Volt I/O, lo cual lo convertiría en un estándar para los jugadores que quisieran incorporarse a la comunidad en toda su extensión.

Fue así como Re-Volt I/O, entre finales del 2015 y mediados del 2017, logró consolidarse y llegar a más jugadores que nunca, formando una comunidad activa de amantes del juego quienes, espontáneamente, se reunían a jugar en línea durante la semana utilizando un paquete de contenido adicional para RVGL, el cual todos debían descargar e instalar por separado para poder jugar. Eventualmente, estas partidas en línea adquirieron un horario definido con fechas y horas acordadas con antelación, para así facilitar la asistencia de los jugadores a los eventos de carreras.

En la actualidad, Re-Volt I/O sigue siendo la comunidad de Re-Volt más grande en términos de jugadores y escala, pero en si todas las comunidades de Re-Volt están unidas y se ayudan unas con otras. Después de todo, se trata de un juego nicho, en donde todos intentan hacerlo accesible y fácil de entender para quienes deseen formar parte de su comunidad.

\section{Re-Volt America}
Si bien Re-Volt I/O fue, durante muchos años, la única comunidad grande de Re-Volt a nivel mundial, no fue mucho después de su gran auge que comenzarían a formarse los demás grupos que, a día de hoy, tienen gran relevancia en la escena multijugador de Re-Volt y que, además, cuentan con un numeroso público y gran actividad. Dentro de estas nuevas comunidades se encuentra Re-Volt America, la comunidad de Re-Volt que abarca a todos los jugadores del continente americano, especialmente de latinoamérica. El logotipo oficial de Re-Volt America, o RVA para abreviar, puede apreciarse en la figura FIGNUM.

\includegraphics[width=10cm, height=10cm]{img/rva.png}

La comunidad de Re-Volt America es concebida originalmente en el año 2017, bajo el nombre de Re-Volt Tournament. No fue hasta después de un par de años que esta sería renombrada a Re-Volt America, debido a la procedencia de sus jugadores, la cual era tanto de norte america como de sudamerica.

En el presente año 2023, Re-Volt America cuenta con una gran cantidad de jugadores activos, y con un sistema de puntuación único en la escena de Re-Volt y sus comunidades en línea. Este complejo sistema de puntuación, y su funcionamiento sostenido durante los últimos 6 años, son la base del problema que busca solucionar este proyecto de título. Con el pasar del tiempo, este sistema se ha convertido en algo muy difícil de mantener para los administradores de la comunidad, tanto a nivel logístico como técnico.

A continuación, se presentaría el contexto del problema en detalle.

\section{Contexto del Problema}
Como ya se mencionó anteriormente, Re-Volt es un videojuego de carreras el cual, gracias al surgimiento de RVGL y sus comunidades impulsoras, es jugado mayoritariamente en línea. Pero, ¿a qué nos referimos con ''jugar en línea''?. Para poder entender este concepto, tenemos que ir a lo que es una carrera en términos conceptuales, y las implicaciones que estas conllevan dentro de un contexto competitivo.

Para poder jugar en línea, cada jugador debe elegir un nombre de usuario, el cual puede incluso variar de partida en partida. Esto se hace una vez que ingresa al juego y avanza en el menú hasta llegar al selector de nombre de usuario en forma de neumático. El nombre que el jugador ingrese aquí será el nombre de usuario con el que se identificará a la hora de ser ingresado a los resultados de cada carrera en la que participe. El selector de nombre de usuario se puede apreciar a continuación en la figura FIGNUM.

\includegraphics[width=15cm, height=8cm]{img/username.png}

Las siguientes dos ilustraciones presentan las instancias clave dentro del juego. En la ilustración FIGNUM, se puede apreciar la perspectiva del jugador al momento de jugar Re-Volt. Luego, en la ilustración FIGNUM, se puede ver la tabla de resultados que se muestra por pantalla a medida que los corredores finalizan la carrera.

\includegraphics[width=15cm, height=8cm]{img/gameplay.jpg}

\includegraphics[width=15cm, height=8cm]{img/results.png}

Normalmente, en las partidas multijugador, suelen jugarse muchas carreras de manera consecutiva. A estas series de partidas en línea se les conoce como ''sesiones''. Cada sesión de Re-Volt consiste en una cantidad predefinida carreras en pistas determinadas, con cierta clase de autos. Estas determinaciones las realiza el anfitrión de la partida, quien las comunica públicamente de manera oportuna para que todos aquellos que deseen participar tengan en cuenta todas las características de la sesión que van a jugar.

Históricamente, Re-Volt America se ha dedicado a organizar sesiones multijugador de Re-Volt para su público, así como también se ha preocupado de llevar la cuenta de los resultados de cada carrera que se ha jugado en ellas mediante un sistema de temporadas. Todos los jugadores que, en algún momento, han participado en las sesiones online organizadas por Re-Volt America, han sido indexados en la base de datos de jugadores que mantiene la comunidad. De esta forma, sus victorias, puntos y otras estadísticas asociadas han sido preservadas a lo largo del tiempo.

Para conseguir un sistema atractivo para sus jugadores, Re-Volt America organiza sus sesiones en una serie de rankings, los cuales consisten en 28 sesiones multijugador cada uno, correspondiendo cada sesión a un día del calendario. Al cumplirse 6 de estos rankings, se completa lo que en RVA se conoce como una temporada. Cada año, en promedio, se inician dos temporadas, y cada una es nombrada a en base a su año de inicio y el periodo en relación a otras temporadas. Por ejemplo, si en el año 2023 se da inicio a una temporada en Enero, estamos hablando de la temporada ''2023'', ya que esta terminará dentro del mismo año en el que se inició. Por otra parte, si la temporada inicia en Diciembre de 2023, eso quiere decir que terminará en 2024, por lo que pasaría a llamarse ''2023-24''.

Es así como Re-Volt America, desde el año 2017 hasta el presente, ha conseguido consolidarse como la comunidad de Re-Volt predilecta para los jugadores tanto de norteamerica como América latina.

\section{Problemática Actual}
De acuerdo a lo que se mencionó en los puntos anteriores, Re-Volt America ha mantenido los registros históricos de las sesiones celebradas a diario de manera interna hace aproximadamente 5 años. A partir de estos registros, se han publicado resultados por sesión y rankings acumulados. Además de esto, RVA también se ha encargado de recoger otro tipo de estadísticas e información de sus jugadores, como pueden ser su nacionalidad, total de puntos acumulados por temporada, carreras corridas, porcentaje por participación y puntajes totales dentro del sistema de puntuación.

En RVGL, cada carrera consiste en que los jugadores corren en una pista y, al final, cada uno termina en una posición dependiendo de quien llega primero a la meta, como en cualquier juego de carreras tradicional. Cada sesión organizada por RVA consiste en 20 carreras que se juegan en 20 pistas diferentes, por lo que en una sesión se producen 20 sets de resultados de carreras.

Para poder llevar una cuenta fiable de todas las carreras jugadas, RVGL cuenta con una funcionalidad que permite obtener un registro escrito de los resultados de cada carrera. Este registro viene en forma de un  archivo separado por comas, el cual es conocido por la comunidad como ''Session Log''. Este es el archivo que utilizan los organizadores y administradores de RVA para calcular los resultados oficiales.

El sistema interno de RVA asigna un puntaje por posición final de cada jugador en una carrera de la siguiente forma:

\includegraphics[width=15cm, height=8cm]{img/points.png}

\includegraphics[width=15cm, height=8cm]{img/points-big.png}

Dichos puntajes, al final de cada sesión, son sumados para obtener un total, normalizados y sometidos a diferentes procesos como la división por posición promedio, multiplicación por porcentaje de participación, entre otras operaciones que ayudan a determinar los resultados finales en el formato de RVA. Dichos resultados cuales terminan viéndose como se muestra en la ilustración 4.

Ilustración 4: Tabla de resultados de RVA.
Todo este proceso de cálculo de resultados es llevado a cabo actualmente de manera semiautomática por los organizadores y administradores de RVA. Este proceso consiste en los siguientes pasos:
1.	Llevar a cabo la sesión multijugador y obtener el Session Log con los resultados crudos de cada carrera.
2.	Pasar dicho Session Log por un programa de escritorio que ayuda a calcular los resultados en el formato de RVA.
3.	Con dicho programa de escritorio, exportar los resultados en un formato separado por comas (archivo CSV).
4.	Tomar los resultados exportados y copiarlos manualmente a un documento maestro en Microsoft Excel, en donde los resultados son indexados junto con el resto.
5.	Agregar cierta información de manera manual, como la fecha de la sesión, número de sesión y corregir nombres de jugadores inválidos.
6.	Publicar una fotografía de los resultados traspasados al documento maestro, y otra foto del ranking total de la temporada que también se mantiene actualizado dentro del mismo documento.
A raíz de este largo y tedioso proceso de cálculo y manejo de resultados es que se ha dado origen a este proyecto, como una oportunidad de mejorar dicho proceso, y poder ofrecer así una mejor experiencia de usuario tanto a los jugadores de RVA, como a los organizadores de la comunidad que dedican su tiempo y pasión por este juego a mantener estos registros históricos al día para todos.

\subsection{Diagrama de la Situación en la Actualidad}
COLOCAR UN BPMN

\section{Propuesta de solución}
Debe explicar en términos generales cómo las TIC pueden resolver o mejorar la(s) problemática identificada y quienes serán los usuarios principales, que tecnología se utilizaría para dar soporte a la propuesta.

\section{Soluciones similares disponibles}
A continuación, se describen las soluciones disponibles que pueden ser catalogadas como similares al proyecto que se presenta.

\subsection{Aplicación para Cálculo de Puntos de Re-Volt I/O}
Existe un trabajo similar hace varios años, el cual fue desarrollado por la comunidad europea de RVGL: Re-Volt I/O. Este trabajo es también una aplicación web que permite a los jugadores importar los resultados de las sesiones multijugador y poder visualizarlos dentro de la misma página; sin embargo, dicho proyecto no cuenta con ningún tipo de interconexión entre resultados, lo que quiere decir que cada sesión de carreras publicada en dicho sitio es independiente de otras, por lo cual este proyecto no cuenta con perfiles de usuario, y por ende no permite visualizar estadísticas de ningún tipo, tampoco relacionar tablas de resultados entre sí, y en general fue hecho para ser algo simple y rápido que sirviera para calcular y renderizar resultados de manera oportuna, y no con una visión de persistencia en mente. A continuación, en la ilustración 4 puede apreciarse la tabla de resultados generada por la aplicación web de Re-Volt: I/O.

\subsection{Aplicación para Cálculo de Puntos de RVA}
RVA cuenta con una aplicación de escritorio...

\section{Justificación del Problema}
Actualmente, no existe ninguna plataforma que permita a los usuarios registrarse y visualizar resultados y estadísticas de las sesiones multijugador organizadas por la comunidad. Lo anterior significa que cuando se juegan partidas online no es posible obtener de manera automática rankings, estadísticas por vehículo y menos por usuario, ya que no hay forma de vincular de manera definitiva a los jugadores a través de un perfil dentro del juego. Tal como puede verse en la ilustración 2. Es por ello por lo que un usuario de nuestra comunidad no puede saber cuántas carreras o sesiones ha ganado con cierto auto, o en cierta pista, cómo se compara al resto, qué porcentaje de carreras ha perdido, etc.





% Proyecto
\chapter{Proyecto}

\section{Objetivo General del Proyecto}
Desarrollar una aplicación web para Re-Volt America, la comunidad del continente americano formada alrededor del videojuego de carreras Re-Volt, 1999. Esta aplicación permite centralizar toda la actividad multijugador en el juego de dicha comunidad, dejando a los usuarios crear cuentas en la web con un perfil asociado. En dicho perfil se podrán visualizar resultados oficiales de las sesiones de carreras que se organicen, además de estadísticas personales.

\section{Objetivos Específicos del Proyecto}
1.	Elaborar una propuesta que consiga atender las necesidades y problemas de los usuarios de Re-Volt America en relación con el almacenamiento y visualización de resultados de partidas online, además de proveer visibilidad a la comunidad en general.
2.	Diseñar la solución de software de procesamiento de datos de sesiones multijugador de Re-Volt, creando interfaces que les permitan a los jugadores visualizar resultados oficiales de las sesiones de carreras en línea, además de estadísticas personales. De esta forma, el diseño de la aplicación en su conjunto centralizará toda la actividad de la comunidad en una sola web.
3.	Implementar la aplicación web, la cual permitirá a los organizadores de sesiones de carreras en línea subir y publicar los resultados de dichas carreras, además de realizar el lazo entre jugadores y cuentas de usuario de esta. El software implementado permitirá a su vez procesar dicha información subida a la web, y así poder mostrar a los usuarios finales una vista clara de sus resultados en carreras y las estadísticas personales.

\section{Metodología de Desarrollo}
Luego de analizar a fondo el proyecto, se concluyó que en lo que respecta al problema se cuenta con una experiencia en el área y complejidad altas, pero con un tamaño pequeño, lo cual representa un bajo riesgo en lo que respecta al problema. Por otra parte, en relación con el software se cuenta con una alta experiencia en términos técnicos. Además, el software en si tiene una complejidad y tamaño pequeños, por lo que el riesgo es realmente bajo en lo que a este respecta. En conclusión, el riesgo ponderado entre el problema y el software a desarrollar es muy bajo.
Según la evaluación del proyecto, se concluyó que este tiene un riesgo total asociado muy bajo, por lo que se contaría con la libertad de utilizar cualquier tipo de metodología de desarrollo para llevarlo a cabo. Como contamos con esta libertad de elección, se buscará sacar provecho de ella, y se ha seleccionado una metodología de desarrollo iterativa, a través de la cual se desarrollará en ciclos. Estos ciclos permitirán al software evolucionar a medida que se recibe retroalimentación por parte de los usuarios, corrigiendo errores y mejorando detalles a medida que se progresa en el desarrollo de la aplicación. Según el contexto en el que se desarrollará esta aplicación, los usuarios de la comunidad tendrán una importante incidencia en el testeo y uso diario de la misma, lo cual implica que estarán presentes durante el proceso de implementación de varias de las funcionalidades que se pretenden lograr con esta propuesta. Se sabe también con antelación que los usuarios estarán dispuestos a ayudar con las pruebas y el testeo de la aplicación. Esta opción fue seleccionada ya que:

\begin{enumerate}
	\item Permite más flexibilidad en cuanto a lo que se planea desarrollar como aplicación web, debido a que a medida que avance el proyecto es muy probable que surjan cambios o nuevas ideas a partir de la retroalimentación recibida de la comunidad.
	\item Debido a la naturaleza del proyecto, utilizar una metodología iterativa es muy beneficioso ya que ésta permitirá una constante supervisión de los cambios realizados al software, lo que finalmente se traducirá en menos errores, y un producto final que se ajuste y esté a la altura de las necesidades de la comunidad.
	\item Con una metodología iterativa será posible contar con prototipos que la comunidad pueda comenzar a utilizar, y a su vez adaptarse a las interfaces y funcionalidades de la aplicación mientras esta se encuentra en desarrollo. Esto es muy importante ya que las sesiones multijugador suelen organizarse a diario, por lo que mientras antes se cuente con una solución funcional, antes podrá la comunidad comenzar a llevar un registro histórico de las partidas online que organiza.
\end{enumerate}

\section{Técnicas y Notaciones}
ejemplo: Diagramas de Casos de Uso -notación UML es utilizada para detallar la funcionalidad del software

\section{Estándares de Documentación}
Agregar estándares...

\section{Herramientas, Frameworks y  Lenguajes Utilizados}
\begin{itemize}
	\item Ruby versión 3.2.2: Lenguaje de programación de alto nivel.
	\item HAML: Lenguaje de marcado para la abstracción de HTML.
	\item Ruby on Rails versión 7: Framework para desarrollo de aplicaciones web fullstack.
	\item MongoDB versión 3: Base de datos orientada a documentos JSON.
	\item MongoDBCompass versión 1.39.0: Visor para la base de datos en MongoDB.
	\item Redis versión 7: Programa de almacenamiento en memoria, utilizado para el caché de datos.
	\item RedisInsight versión 2.30.0: Visor para el almacenamiento del caché en Redis.
	\item NodeJS versión 16: Entorno de servidor multiplataforma utilizado para la conversión de archivos en runtime.
	\item Yarn versión 1.22: Gestor de paquetes para JavaScript.
	\item Docker versión 24.0.2: Tecnología que permite crear y utilizar contenedores de Linux. Para efectos de este proyecto fue utilizado con el fin de probar el software desarrollado en la distribución de Linux requerida.
\end{itemize}







% Factibilidad
\chapter{Factibilidad}

\section{Factibilidad Técnica}
Se realizó un estudio previo de factibilidad técnica, el cual arrojó que las tecnologías que se utilizarán para desarrollar el software en cuestión son capaces de resolver el problema.
Descomponiendo los aspectos técnicos, podemos decir que está más que comprobado que Ruby on Rails ha sido utilizado para conseguir resultados similares con proyectos del mismo tipo, por lo que es completamente factible utilizarlo en el contexto de este proyecto.
Las tecnologías utilizadas y la manera en que se montará el proyecto permitirá también que este sea perfectamente escalable y mejorable.
Es por esto que es factible el desarrollo de este proyecto.

\section{Factibilidad Operativa}
La factibilidad operativa es fácilmente demostrada al comparar la solución propuesta con el sistema que pretende reemplazar. Vale decir que actualmente quienes hacen uso del sistema de hojas de cálculo maestras y la aplicación de escritorio para el procesamiento de sesiones actualmente en RVA verían su trabajo facilitado en todos los sentidos al contar con una plataforma que se encargue de llevar la cuenta de todo, procesar los resultados y mantener las tablas de resultados y rankings de temporadas de manera automática.

\section{Factibilidad Económica}
Agregar factibilidad económica...

\section{Conclusión de Factibilidad}
Gracias al análisis realizado en los puntos anteriores, se puede concluir que el proyecto es completamente factible. Podrá ser desarrollado de manera efectiva y esto asegura una mejora sustancial del proceso actual de procesamiento de datos de las sesiones de RVA y manejo de resultados en general.


% Requerimientos del Software
\chapter{Requerimientos del Software}

\section{Límites}
La aplicación no permitirá relacionar automáticamente cualquier nombre de jugador con su perfil dentro de la misma. Esto quiere decir que la aplicación estará limitada a relacionar el nombre que sea subido mediante el Session Log con un perfil que exista previamente en la plataforma.
La aplicación no permitirá el uso de su API de por parte de terceros, lo que quiere decir que la API de RVA estará limitada a la misma aplicación.

\section{Objetivo General del Software}
El sistema manejará información del proceso de cálculo de resultados de carreras para que la comunidad centralice los rankings y estadísticas por usuario dentro del mismo, es decir, reducirá el trabajo manual requerido actualmente para este procesamiento de resultados, y así hará más eficiente todo el proceso que conlleva mantener los rankings actualizados.


\subsection{Objetivos Específicos del Software}
•	El sistema permite que los administradores de la comunidad puedan subir los archivos de resultados generados por RVGL y, de esta forma, los resultados son generados automáticamente dentro de la aplicación. Esto elimina el tiempo de los organizadores de calcular los resultados manualmente.

•	El sistema permite que los usuarios puedan ver sus estadísticas en tiempo real, lo que elimina la necesidad de llevar la cuenta de manera manual por cada uno de ellos.

•	El sistema permite enlazar los nombres de usuario utilizados en las sesiones multijugador de RVGL a perfiles dentro de la aplicación, lo cual hace posible la recopilación y atribución de métricas individuales por jugador, y agiliza la visualización de resultados.

•	El sistema permite llevar un registro histórico de manera automática según se suben y se procesan los archivos de resultados en la aplicación, lo cual elimina completamente la necesidad de la comunidad de mantener toda esta información actualizada de manera manual.


\section{Requerimientos Funcionales del Software}

\begin{center}
	\begin{tabular}{ | l | p{15cm} |}
		\hline
		\multicolumn{2}{|c|}{Módulo de Registros de Autos de Re-Volt America} \\
		\hline
		\multicolumn{1}{|c|}{Id} & \multicolumn{1}{|c|}{Descripción} \\
		\hline
		RF\_01 & La plataforma contará con un módulo de creación de autos. Los autos deben contar con todos los parámetros obligatorios. Sólo los administradores pueden crear autos. \\ \hline

		RF\_02 & La plataforma contará con un módulo de visualización de autos. El listado estará separado por temporadas y por clases de autos. Este módulo estará disponible para cualquier tipo de usuario. \\ \hline

		RF\_03 & La plataforma contará con un módulo de edición de un auto, los cuales deben contar con todos los parámetros obligatorios para ser modificados. Este módulo estará disponible sólo para los administradores. \\ \hline
		
		RF\_04 & La plataforma contará con un módulo de eliminación de un auto. La eliminación no tendrá efectos secundarios a nivel de la base de datos, ya que la aplicación estará preparada para manejar excepciones cuando las entradas de corredores estén asociadas a un auto que no exista. Este módulo sólo estará disponible para administradores. \\ \hline
		
		RF\_05 & La plataforma contará con un módulo de visualización de un solo auto. En esta vista se podrán ver todos los parámetros del auto. Este módulo estará disponible sólo para los administradores. \\ \hline
	\end{tabular}
\end{center}

\end{document}




















